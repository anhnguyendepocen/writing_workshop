\documentclass[]{tufte-book}

% ams
\usepackage{amssymb,amsmath}

\usepackage{ifxetex,ifluatex}
\usepackage{fixltx2e} % provides \textsubscript
\ifnum 0\ifxetex 1\fi\ifluatex 1\fi=0 % if pdftex
  \usepackage[T1]{fontenc}
  \usepackage[utf8]{inputenc}
\else % if luatex or xelatex
  \makeatletter
  \@ifpackageloaded{fontspec}{}{\usepackage{fontspec}}
  \makeatother
  \defaultfontfeatures{Ligatures=TeX,Scale=MatchLowercase}
  \makeatletter
  \@ifpackageloaded{soul}{
     \renewcommand\allcapsspacing[1]{{\addfontfeature{LetterSpace=15}#1}}
     \renewcommand\smallcapsspacing[1]{{\addfontfeature{LetterSpace=10}#1}}
   }{}
  \makeatother

\fi

% graphix
\usepackage{graphicx}
\setkeys{Gin}{width=\linewidth,totalheight=\textheight,keepaspectratio}

% booktabs
\usepackage{booktabs}

% url
\usepackage{url}

% hyperref
\usepackage{hyperref}

% units.
\usepackage{units}


\setcounter{secnumdepth}{2}

% citations
\usepackage{natbib}
\bibliographystyle{apalike}

% pandoc syntax highlighting

% longtable
\usepackage{longtable,booktabs}

% multiplecol
\usepackage{multicol}

% strikeout
\usepackage[normalem]{ulem}

% morefloats
\usepackage{morefloats}


% tightlist macro required by pandoc >= 1.14
\providecommand{\tightlist}{%
  \setlength{\itemsep}{0pt}\setlength{\parskip}{0pt}}

% title / author / date
\title{Writing and Rewriting for Scientific Communication}
\author{Brooke Anderson}
\date{March 28, 2019}

\usepackage{booktabs}
\usepackage{amsthm}
\usepackage{fontspec}
    \setmainfont{Gill Sans}
\makeatletter
\def\thm@space@setup{%
  \thm@preskip=8pt plus 2pt minus 4pt
  \thm@postskip=\thm@preskip
}
\makeatother

\begin{document}

\maketitle



{
\setcounter{tocdepth}{1}
\tableofcontents
}

\hypertarget{introduction}{%
\chapter{Introduction}\label{introduction}}

\newthought{Welcome to our one-week} writing workshop! This workshop is
meant for students who are working on a manuscript that they plan to submit for
peer review at a scientific journal. We'll be focusing on how to structure and
rewrite your manuscript to make it clearer and easier to read.

There are a number of things that are important that we won't cover here:

\begin{enumerate}
\def\labelenumi{\arabic{enumi}.}
\tightlist
\item
  \textbf{Grammar and usage.} These are both very important, and you'll want these
  to be impeccable when you submit your manuscript. However, there are loads of
  resources and courses for this, and while I often see cases where students
  haven't mastered this yet, I don't think it's hard to find the resources to
  learn these. Also, at least through your graduate career, you will almost always
  have co-authors, including a senior author (typically the head of your research
  group). Grammar and usage problems are easy to fix quickly when your senior author
  edits your paper. The things we will cover here take more time and work.
\item
  \textbf{Doing a literature review.} This is an art of its own. It's very important
  for your manuscript to use good examples from the literature to set up why your
  study is important and to put what you find in the context of previously
  published studies. If you are pursuing a graduate degree, you will need to
  explore the literature enough to understand the current state of science in your
  field and how your work fits into the field. With certain types of manuscripts,
  like commentaries and reviews, the whole manuscript might be a synthesis and
  exploration of the literature. Ideally, you will have done quite a bit of
  work to explore and get examples from the literature by the time we do this
  workshop, as this will provide some of the raw material for your manuscript.
  However, we will only tangentially discuss how to explore and review the literature.
  This will be in the context of building and presenting arguments in your manuscript,
  as evidence from the literature will often be needed to do this, particularly in the
  Introduction and Discussion.
\item
  \textbf{Creating effective figures and tables.} This is also extremely important.
  At worst, bad figures and tables can make a manuscript look unprofessional. If
  they do, then a reviewer may worry that the research presented in the study is
  similarly sloppy or poorly done. Figures and tables are the best chance to immediately
  engage a reader, and they'll often stick with the reader, if they're good, much
  longer than the text. They also serve double-duty, as you'll use
  them in presentations you make about the research. Finally, a really good figure or
  table encourages the reader to explore the results from the research deeply, and
  start looking for patterns or trends that suggest ideas for next steps for the
  research area. Manuscripts are more likely to get cited, and to have a higher
  impact on your field, if readers leave the paper with ideas of research they want
  to do to take the next steps. There are some excellent resources for learning how
  to design and create figures and tables. I particularly like books by Edward Tufte
  and Howard Wainer. There are also a number of books on visualization that are
  specific to a programming language, like R or Python.
\item
  \textbf{How to get words on paper.} Our focus will be on improving and
  revising a manuscript once you've already got some stuff down on paper. We won't
  talk much about how to get the stuff down in the first place. There are some
  good books on the topic of writing initial drafts. Different writers have
  different styles for this. Some will work very hard on the first draft, so that
  rewriting doesn't need to be extensive. I tend to start by writing a lot, but
  all of it pretty messy, and then wrangling and editing that raw
  material into a more solid draft. I usually do lots of rounds of drafts,
  first moving big parts around to form the larger structure of
  the paper and to see where there are holes that need more evidence or support.
  Then, in later drafts, my editing process involves drilling down to make
  the writing clearer and easier to read. In this workshop we'll focus on the work
  involved with the later stages of editing. If you'd like to learn more about
  tips and techniques for earlier stages, where the focus is generating your
  starting material, I like the books of Peter Elbow. Stephen King's \emph{On Writing}
  is also great, particularly in inspiring you to get something down even when
  you're discouraged (by bad reviews or by how big the task seems, for example).
\end{enumerate}

Click on the \textbf{Next} button (or navigate using the
links at the top of the page) to continue.

\hypertarget{readings-for-the-workshop}{%
\section{Readings for the workshop}\label{readings-for-the-workshop}}

We will be reading and trying out the ideas from several books for this
workshop. They all have connections to Joseph Williams \emph{Style} book, which has
gone through several editions (and titles) since it was first published in the
1980s.

There are a lot of good writing books out there, but this is by far my favorite.
I regularly pick it back up when I am editing manuscripts, particularly when I have
a section of a draft that I am struggling to write clearly. What I like about it
is that it focuses on how we can edit our writing to make it easier for readers
to read and understand. It bases its advice on how readers process information as
they read, and it gives advice that is easy to use to diagnose problems and
fix them.

I cannot overstate how important it is to edit your scientific manuscript until
it can be read easily (and feel ``quick'' to read) and until it is hard to
misunderstand. You are moving into work where you will be the expert in the
room. You will know your very specific topic better than all but a handful of
people in the world, and often better than the editor and reviewers
for your manuscript. You are teaching through your manuscript, and the text
must be clear for readers to learn what you've learned.

Also, scientific papers are hard work to read, even when they are written
beautifully. As the reader, you must work to understand the science and
statistical methods that were used to generate results for the paper. Your mind
works hard to integrate the paper's results with what you know from other papers
in the field. If the writer does not make the writing clear and easy to read, it
can feel impossible to read the paper, because your brain can't simultaneously
try to figure out what the writer was trying to say and what the science in the
paper is saying. You have worked hard on your science. It's worth the effort to
edit your manuscript until you remove all overhead and make the writing easy to
read.

Most people, even senior scientists, write first drafts that are a
drag to read. Even if all the information is there, it's a lot of work to
mentally extract it, and it's often hard to follow along with the arguments and
explanations. It's fine for your early drafts to be like that. It's not good for
the submitted manuscript to still feel so heavy.

The examples in \emph{Style} are mainly from academic writing, although not
specifically from scientific manuscripts or manuscripts meant for peer review.
Fortunately, two other writers have taken Williams's ideas and illustrated
how they work in scientific writing. Further, they give some tips specific to
scientific writing, including how to plan and structure certain sections---like the
Introduction and Discussion. These are the second and third books on the list.

The fourth book in the list includes a section on arguments. We often don't think of
scientific writing as argumentative, but it really is. You will need to convince
readers that your study is important, that it advances the science. Then you'll
need to not just present your results, but provide an interpretation for those
results. You will need to walk your readers through why those interpretations
are reasonable and appropriate. While many students come into graduate school
with a reasonable mastery of grammar, they often present their arguments in a
way that is scattered and diffuse. Often, when I am editing manuscripts with
first authors who are students, much of the editing process is related to
finding and fleshing out their arguments. We won't be able to go deeply into
rhetoric in this workshop, but we will cover the basics of building arguments
in your writing based on the advice in this book.

The full list of books for the workshop is:

\begin{enumerate}
\def\labelenumi{\arabic{enumi}.}
\tightlist
\item
  \href{https://www.amazon.com/Style-Basics-Clarity-Grace-2nd/dp/0321330854/ref=sr_1_20?dchild=1\&keywords=joseph+williams+style\&qid=1594599265\&s=books\&sr=1-20}{\emph{Style: The Basics of Clarity and Grace (2nd edition)} by Joseph
  Williams.}
  (This edition is no longer in print. I prefer the earlier editions, which are
  shorter. There tend to always be used copies available on Amazon. There also
  seems to be a pdf of an early edition version \href{https://sites.duke.edu/niou/files/2014/07/WilliamsJosephM1990StyleTowardClarityandGrace.pdf}{available
  online}
  that you can use before your print edition arrives.)
\item
  \href{https://www.amazon.com/Writing-Science-English-Chicago-Publishing/dp/022602637X/ref=sr_1_2?crid=3R5FVEBHLEE6M\&dchild=1\&keywords=writing+science+in+plain+english\&qid=1594599082\&sprefix=writing+science+in+p\%2Caps\%2C185\&sr=8-2}{\emph{Writing Science in Plain English} by Anne E.
  Greene.}
\item
  \href{https://www.amazon.com/Writing-Science-Papers-Proposals-Funded/dp/0199760241/ref=pd_sbs_14_2/140-7397953-6612456?_encoding=UTF8\&pd_rd_i=0199760241\&pd_rd_r=a18d3605-570b-4f9c-a9b7-560aab4c2f19\&pd_rd_w=reAtV\&pd_rd_wg=ldykT\&pf_rd_p=bdc67ba8-ab69-42ee-b8d8-8f5336b36a83\&pf_rd_r=6JENX0H17SA0SEH0B3SH\&psc=1\&refRID=6JENX0H17SA0SEH0B3SH}{\emph{Writing Science} by Joshua
  Schimel}
\item
  \href{https://www.amazon.com/Research-Chicago-Writing-Editing-Publishing/dp/022623973X/ref=sr_1_2?dchild=1\&keywords=craft+of+research\&qid=1594599147\&s=books\&sr=1-2}{\emph{The Craft of Research (4th edition)} by Wayne C. Booth, Gregory C.
  Colomb, Joseph M. Williams, Joseph Bizup, and William T.
  FitzGerald.}
\end{enumerate}

We will also use three published articles as examples. None of these are traditional research
articles, but rather cover a range of commentary/editorial-style articles. All three should be
available either directly online or through CSU. Please let me know if you have any problem
accessing any of these articles.

\begin{enumerate}
\def\labelenumi{\arabic{enumi}.}
\tightlist
\item
  \href{https://journals.ametsoc.org/bams/article/90/6/799/59608/When-Do-Losses-Count-Six-Fallacies-of-Natural}{Gall, Melanie, Kevin A. Borden, and Susan L. Cutter. ``When do losses count?
  Six fallacies of natural hazards loss data.'' Bulletin of the American
  Meteorological Society 90.6 (2009):
  799-810.}
\item
  \href{https://journals.plos.org/ploscompbiol/article?id=10.1371/journal.pcbi.1007513}{Cheplygina, Veronika, et al.~``Ten simple rules for getting started on
  Twitter as a scientist.'' PLoS Computational Biology 16.2 (2020):
  e1007513-e1007513.}
\item
  \href{https://ajph.aphapublications.org/doi/abs/10.2105/AJPH.84.5.819}{Schwartz, Sharon. ``The fallacy of the ecological fallacy: the potential
  misuse of a concept and the consequences.'' American Journal of Public Health
  84.5 (1994):
  819-824.}
\end{enumerate}

I have picked these papers because they all provide important ideas and advice, and
they all have some strong writing. In particular, they have a clear overarching structure
and state and defend their arguments well. Also, like most published papers (mine included),
they include examples where sentences or paragraphs could be edited to be easier to read
and harder to misunderstand. They therefore provide some nice examples of both the good and
the bad (none get into ``the ugly'', but you won't have to look too far in the scientific
literature to find some that do), and so we can see that even very good papers have room
for further editing and try our hand at doing that.

\hypertarget{workshop-schedule}{%
\section{Workshop schedule}\label{workshop-schedule}}

Each day of the week, we will focus on a different topic.

Schedule:

\begin{itemize}
\tightlist
\item
  \textbf{Day 1:} Sentences
\item
  \textbf{Day 2:} Openings
\item
  \textbf{Day 3:} Paragraphs
\item
  \textbf{Day 4:} Arguments
\item
  \textbf{Day 5:} Words
\end{itemize}

You can navigate to the material for each day through the table of contents links
above.

\hypertarget{day-1-sentences}{%
\chapter{Day 1: Sentences}\label{day-1-sentences}}

\newthought{We will cover several different} techniques for editing and rewriting
your manuscript, and they will target different parts (e.g., Introduction, Results) or
levels (e.g., paragraphs, sentences) of your manuscript. Some will take a lot of mental
work and editing to put in place. We will start, however, with some techniques that
are easy to do, and make a huge difference in how easy your manuscript is to read, but
that are underused in science. These techniques focus on editing at the sentence level.

To learn these techniques, you will be reading several chapters of the Joseph
Williams book (\emph{Style}). If you haven't received that one yet, you can find link
to an online scan of an earlier version (like the one I've sent you)
\href{https://sites.duke.edu/niou/files/2014/07/WilliamsJosephM1990StyleTowardClarityandGrace.pdf}{here}.

The techniques are revisited in the other books I sent, all of which are built on
the ideas that \emph{Style} presents. I'll include some suggestions for additional readings
from some of the other books, and you can use that if you feel like you haven't mastered
the ideas yet from reading the \emph{Style} chapters (or if you're so excited about them you
want to see more examples!). If you'd like examples from scientific writing, you can find
them in the suggested additional reading from \emph{Writing Science in Plain English} and
\emph{Writing Science}.

\begin{enumerate}
\def\labelenumi{\arabic{enumi}.}
\tightlist
\item
  \textbf{Read Chapter 1 from \emph{Style}.}
\end{enumerate}

This first chapter sets up Joseph Williams's goals and philosophy for the book,
and it captures nicely what we'll be trying to do through our workshop this
week. Notice in particular that we won't be focusing on ``correct'' or grammatical
writing, but instead on editing to make things easier for our readers.

\begin{enumerate}
\def\labelenumi{\arabic{enumi}.}
\setcounter{enumi}{1}
\tightlist
\item
  \textbf{Exercise: Read and rank the three example papers}
\end{enumerate}

In the
\href{https://geanders.github.io/writing_workshop/1-1-readings-for-the-workshop.html\#readings-for-the-workshop}{Introduction},
I've included links to three papers that we'll use as examples, in addition to
working with your own manuscripts. Read these three papers and rank them (1 as
best to 3 as worst) in terms of: (1) how easy you found it to read; (2) how much
you enjoyed reading it; (3) how much you learned by reading it.

\begin{enumerate}
\def\labelenumi{\arabic{enumi}.}
\setcounter{enumi}{2}
\tightlist
\item
  \textbf{Read Chapter 2 from \emph{Style} (Chapters 3 and 4 in the second edition).}
\end{enumerate}

This chapter explains how you can think of sentences in terms of characters and
their actions, and how you can diagnose if a sentence is not divided into
subject and verb in a way that highlights these elements. While it may seem that
this idea would be more important in fiction, this chapter provides examples of how
it can clarify sentences in academic writing.

\emph{Additional reading.} If you would like more information on the ideas in this
chapter of \emph{Style}, you can also check out Chapter 3 from \emph{Writing Science in Plain English}
and Chapter 17.2 from \emph{The Craft of Research}.

\begin{enumerate}
\def\labelenumi{\arabic{enumi}.}
\setcounter{enumi}{3}
\tightlist
\item
  \textbf{Identify the characters in the example papers.}
\end{enumerate}

In the three example papers, re-read the Introductions. List three to five main
characters in the Introduction of each paper, using the explanation of ``main characters''
that Joseph Williams provides in Chapter 2 of \emph{Style}. Remember that, in some cases,
characters can be concepts.

Next, identify the characters in the following sentence:

\begin{quote}
``The use of the ecological fallacy to explain the discrepancy between
individual and ecological correlations may have unintended consequences.''
Schwartz, 1994
\end{quote}

Try to rewrite the sentence to conform with the first two principles of clear
writing (p.~21 of the online version of \emph{Style}).

Do the same with the following sentence (it may help to replace the semi-colon
with a period and diagnose / fix the resulting two sentences separately):

\begin{quote}
``For these reasons, using Twitter appropriately can be more than just a social
media activity; it can be a real career incubator in which researchers can
develop their professional circles, launch new research projects and get helped
by the community at various stages of the projects.'' Cheplygina et al., 2020
\end{quote}

\begin{enumerate}
\def\labelenumi{\arabic{enumi}.}
\setcounter{enumi}{4}
\tightlist
\item
  \textbf{Identify the characters in a paragraph of your manuscript.}
\end{enumerate}

Pick a paragraph of your manuscript. List three to five main characters in that paragraph.
In the sentences in these paragraphs, are the main characters the subjects of the
sentences? Are there any examples where the action is in abstract nouns (e.g., knowledge,
determination) rather than in verbs (e.g., know, determine)?

Pick one sentence of this paragraph. Rewrite it two ways. First, rewrite it in a way that
violates the first two principles of clear writing (p.~21 of the online version of \emph{Style})
and then in a way that conforms with these principles.

Next, read through the full paragraph and try to diagnose every sentence that is
problematic, from this perspective. Use the following advice from the \emph{Style}
book:

\begin{quote}
``A quick method is simply to run a line under the first five or six words of
every sentence. If you find that (1) you have to go more than six or seven words
into a sentence to get past the subject to the verb and (2) the subject of the
sentence is not one of your characters, take a hard look at that sentence; its
characters and actions probably do not align with subjects and verbs.''
\end{quote}

If you do not find any in this paragraph, look through other paragraphs in your
manuscript. Write down up to three sentences that you find, and we'll work together
to diagnose if they do indeed have this problem and, if so, fix them.

\begin{enumerate}
\def\labelenumi{\arabic{enumi}.}
\setcounter{enumi}{5}
\tightlist
\item
  \textbf{Identify and replace nominalizations.}
\end{enumerate}

In Chapter 2 of \emph{Style}, Joseph Williams defines \emph{nominalizations} and explains how
they can confound clarity in a sentence. Identify any nominalizations in the following
sentence:

\begin{quote}
``We begin with a description of the validity framework and the definition of
key terms.'' Schwartz, 1994
\end{quote}

Rewrite the sentence without any nominalizations.

Next, find three sentences from your own manuscript with nominalizations. There
are some tips in the section in Chapter 2 on ``Looking for Nominalizations''.
Write these down, and then try rewriting them to remove the nominalizations.

\begin{enumerate}
\def\labelenumi{\arabic{enumi}.}
\setcounter{enumi}{6}
\tightlist
\item
  \textbf{Editing sentences in a paragraph.}
\end{enumerate}

Using all the techniques from Chapter 2, diagnose problems in the following
paragraph and rewrite it:

\begin{quote}
``Natural hazard losses exhibit an upward trend over time. This is a function
of increases in wealth and population but is also attributed to better loss
accounting in recent years. The escalating pattern of hazard losses is therefore
partially an artifact of advances in reporting losses, but how much or how
little this effect contributes to the skyrocketing losses in comparison to
effects of population growth and increasing wealth in high hazard areas is
unclear.'' Gall et al., 2009
\end{quote}

\textbf{All the following steps are things for you to try after our workshop
meeting on ``Sentences''.}

\begin{enumerate}
\def\labelenumi{\arabic{enumi}.}
\setcounter{enumi}{7}
\tightlist
\item
  \textbf{Read Chapter 3 from \emph{Style} (Chapter 5 of the second edition).}
\end{enumerate}

This chapter takes the next step---it starts considering how sentences fit
within a paragraph, and how ordering the information in the sentence can
help the flow across the paragraph.

\emph{Additional reading.} If you would like more information on the ideas in this
chapter of \emph{Style}, you can also check out Chapter 7 from \emph{Writing Science in
Plain English}, Chapter 12 in \emph{Writing Science}, and Chapter 17.3 from
\emph{The Craft of Research}.

\begin{enumerate}
\def\labelenumi{\arabic{enumi}.}
\setcounter{enumi}{8}
\tightlist
\item
  \textbf{Identify topics in a paragraph of your manuscript.}
\end{enumerate}

Pick a paragraph of your manuscript. Identify all the topics in the paragraph.
Is there a consistent topic string, as defined in Chapter 3 of \emph{Style}?
Are your topics visible? Revise your paragraph based on this diagnosis.

Next, read through all the paragraphs in your manuscript. Highlight the topics
throughout. Are there certain sections where it looks like the topics might not
be very visible, or where there may be too many topics in a paragraph?

\begin{enumerate}
\def\labelenumi{\arabic{enumi}.}
\setcounter{enumi}{9}
\tightlist
\item
  \textbf{Read Chapter 4 from \emph{Style} (Chapter 6 of the second edition).}
\end{enumerate}

This chapter discusses strategies for ending sentences, to emphasize the point
you want to emphasize. This chapter is helpful, but you'll get more immediate
milage from mastering the content in Chapters 2 and 3. For now, read through
this chapter so you can mull it over, but I don't have any exercises for you
based on it.

\hypertarget{day-2-openings}{%
\chapter{Day 2: Openings}\label{day-2-openings}}

\newthought{On Day 1, we focused on mechanics} rather than content. Today,
we'll talk about content, focusing on your Introduction. I usually find this
section the hardest to write, and I'll often put it off until last. You need to
draw in your reader and convince him or her that the rest of the paper is worth
reading (and, for a research article, that the science was worth doing). If a
reviewer is going to hate your paper, he or she often decides by the end of the
Introduction (based on review comments I've gotten for past papers). And while
the other reviewers won't decide that they love it by that point, they can be
convinced that they are in reasonably good hands, and that you are competent in
the area and familiar with the field and open problems in it.

\emph{Writing Science} discusses the job of an Introduction in a scientific paper,
and how an Introduction can fail. For today's meeting, you'll be reading through
several chapters in this book, evaluating whether and how the Introductions in
the three example papers do their job, thinking about how to frame the
Introduction for your manuscript.

\begin{enumerate}
\def\labelenumi{\arabic{enumi}.}
\tightlist
\item
  \textbf{Summarize your manuscript.}
\end{enumerate}

In 2--3 sentences, describe what your manuscript covers.

In 2--3 more sentences, explain why your manuscript is important in advancing
science in your field.

\begin{enumerate}
\def\labelenumi{\arabic{enumi}.}
\setcounter{enumi}{1}
\tightlist
\item
  \textbf{Read Chapter 4 of \emph{Writing Science}.}
\end{enumerate}

This chapter explains common story structures, and how they can be applied
to scientific writing. Near the end of this chapter (section 4.3), it presents
the key components of a paper's Introduction, Methods, Results, and Discussion.

\begin{enumerate}
\def\labelenumi{\arabic{enumi}.}
\setcounter{enumi}{2}
\tightlist
\item
  \textbf{Identify key Introduction components in an example paper.}
\end{enumerate}

Re-read the Introductions of the example papers by Cheplygina et al.~and
Schwartz. Try to identify the following components in the Introduction:

\begin{itemize}
\tightlist
\item
  What is the larger problem that the paper will be tackling?
\item
  What is the relevant context for that problem?
\item
  What are the key characters?
\item
  What background information is given to help the reader understand
  the specific work in the paper?
\item
  What is the ``challenge'' (``the specific hypotheses/questions/goals of
  the current work'')?
\end{itemize}

Go through each of these two Introductions and use three different colors to
highlight elements that you think are part of the Opening, the Background, and
the Challenge, based on how those parts are explained in this chapter. (You will
be reading other chapters today that go deeper into these ideas, so you may come
back and revise your answers based on your later reading.)

\begin{enumerate}
\def\labelenumi{\arabic{enumi}.}
\setcounter{enumi}{3}
\tightlist
\item
  \textbf{Read Chapter 5 of \emph{Writing Science}.}
\end{enumerate}

This chapter covers the very first part of the Introduction, which the author
calls the ``Opening''.

\begin{enumerate}
\def\labelenumi{\arabic{enumi}.}
\setcounter{enumi}{4}
\tightlist
\item
  \textbf{Diagnose the Opening for an example paper.}
\end{enumerate}

Re-read the first paragraph of the Schwartz paper. Based on this paragraph, who
do you think is the intended audience for the paper? What is the larger issue
the paper will address? Do you think that they are properly ``advertising'' what
they will later cover in the paper?

\begin{enumerate}
\def\labelenumi{\arabic{enumi}.}
\setcounter{enumi}{5}
\tightlist
\item
  \textbf{Define Opening components for your paper.}
\end{enumerate}

Write down the following for your own manuscript:

\begin{itemize}
\tightlist
\item
  What is the target audience? Is it broad / interdisciplinary or targeted to
  researchers in a certain field?
\item
  What is the larger issue that the manuscript will address? In writing this, be
  sure to be clear about the scope with which the manuscript will cover this
  issue.
\item
  What are a few elements of the issue that are interesting but that your
  manuscript will \textbf{not} address?
\end{itemize}

\begin{enumerate}
\def\labelenumi{\arabic{enumi}.}
\setcounter{enumi}{6}
\tightlist
\item
  \textbf{Revise the Opening for your manuscript.}
\end{enumerate}

Take a look at the first paragraph of your Introduction in the context of your
answers to the questions in the previous prompt. Does your Opening need the
whole paragraph, or just the first sentence or two? Does this first paragraph
include any signals to clarify what audience you expect for the manuscript? Does
the paragraph give readers clues on the larger issue that the manuscript will
address? Does the paragraph ``overpromise'', indicating that the manuscript will
cover a larger scope than it does?

Write a revised version of your first paragraph where you address any
limitations of your previous draft of your Opening.

\textbf{All the following steps are things for you to try after our workshop
meeting on ``Openings''.}

\begin{enumerate}
\def\labelenumi{\arabic{enumi}.}
\setcounter{enumi}{7}
\tightlist
\item
  \textbf{Read Chapters 6 and 7 of \emph{Writing Science}.}
\end{enumerate}

Chapter 6 covers how to move from the Opening of your Introduction to the
Challenge that you present before moving into other parts of your paper. Chapter
7 discusses how to end an Introduction with a ``Challenge,'' motivating everything
you present in the rest of the manuscript.

\begin{enumerate}
\def\labelenumi{\arabic{enumi}.}
\setcounter{enumi}{8}
\tightlist
\item
  \textbf{Evaluate the Funnel and Challenge of example papers.}
\end{enumerate}

Re-read the Introductions for the three example papers. Find and write down the
text in each paper that presents the Challenge of the paper.

Next, re-read the Introduction of the Gall et al.~paper. How does the
Introduction funnel the readers from the Opening to the Challenge at the end of
the Introduction? What background information does it give readers to help them
understand and appreciate the Challenge? What examples do they use from the
literature to convince readers that their Challenge is important?

\begin{enumerate}
\def\labelenumi{\arabic{enumi}.}
\setcounter{enumi}{9}
\tightlist
\item
  \textbf{Define the Challenge of your manuscript.}
\end{enumerate}

In 1--2 sentences, write the Challenge of your manuscript.

Look at the draft of your Introduction. Is this Challenge clear in the text? If
so, is it at the end of the Introduction, or somewhere else?

Write down three pieces of background information that you think it is critical
for your reader to know to believe that your manuscript's Challenge is
important. Draft a paragraph one each of these explaining and presenting the
background information to readers.

Write down two examples that you think might help readers understand why your
Challenge is important. Draft a few sentences presenting each of these examples.

Revise your Introduction to integrate your drafted material on the background
and examples.

\begin{enumerate}
\def\labelenumi{\arabic{enumi}.}
\setcounter{enumi}{10}
\tightlist
\item
  \textbf{Read Chapters 8 and 9 of \emph{Writing Science}.}
\end{enumerate}

These next two chapters move into advice on writing the body of the manuscript
(the ``Action'', which often includes Methods and Results) and the ``Resolution''
(typically the Discussion). Think about what you're hoping to achieve through
these later parts of your manuscript.

\hypertarget{day-3-paragraphs}{%
\chapter{Day 3: Paragraphs}\label{day-3-paragraphs}}

\newthought{Today, we'll be moving back} from content to mechanics. Today,
we'll talk about editing for clarity at a different level, the level of
paragraphs. We will look at how we can move around text in a paragraph to make
it easier for a reader to quickly understand what they should get out of a
paragraph and to easily follow ideas in the paragraph.

To learn these techniques, you will be reading several chapters of the Joseph
Williams book (\emph{Style}). If you haven't received that one yet, you can find link
to an online scan of an earlier version (like the one I've sent you)
\href{https://sites.duke.edu/niou/files/2014/07/WilliamsJosephM1990StyleTowardClarityandGrace.pdf}{here}.

The techniques are revisited in the other books I sent, all of which are built
on the ideas that \emph{Style} presents. I'll include some suggestions for additional
readings from some of the other books, and you can use that if you feel like you
haven't mastered the ideas yet from reading the \emph{Style} chapters. If you'd like
examples from scientific writing, you can find them in the suggested additional
reading from \emph{Writing Science in Plain English} and \emph{Writing Science}.

\begin{enumerate}
\def\labelenumi{\arabic{enumi}.}
\tightlist
\item
  \textbf{Reach Chapter 5 of \emph{Style} (online version).}
\end{enumerate}

This chapter discusses how to structure paragraphs so they seem to ``hang
together''. It presents techniques for diagnosing and fixing problems with
thematic strings to help make a paragraph more coherent, as well as how to order
the issue and discussion within a paragraph.

\emph{Additional reading.} If you would like more information on the ideas in this
chapter of \emph{Style}, you can also check out Chapter 10 from \emph{Writing Science in
Plain English}.

\begin{enumerate}
\def\labelenumi{\arabic{enumi}.}
\setcounter{enumi}{1}
\tightlist
\item
  \textbf{Identify thematic strings in a paragraph from an example paper.}
\end{enumerate}

Re-read this paragraph from the Cheplygina et al.~paper. When you first read
this paragraph, did you find it easy or hard to read?

\begin{quote}
``Even if you use Twitter only for professional purposes, consider opening up a
little bit to show your followers you are a real person. People outside your
field are not likely to follow you if your tweets are only about sharing events,
articles, and positions in your own field. You need to add an extra
ingredient---your opinions, or something personal---to what you share. One way to do
this is through sharing failures: a rejected paper or job application, or even a
spilt coffee. This is a great way to give and receive moral support from other
academics.''
\end{quote}

With the above paragraph, do the following exercise from Chapter 5 of \emph{Style}
(without re-reading the paragraph!):

\begin{quote}
``Make two lists. In one, list the characters you remember. In the other, list
just two or three words that would capture the central concepts that the writer
weaves around those characters, words that constitute the conceptual center of
that paragraph.''
\end{quote}

Go back and look at the paragraph. Do the thematic strings in the paragraph
agree with the characters and concepts that you remembered and listed?

\begin{enumerate}
\def\labelenumi{\arabic{enumi}.}
\setcounter{enumi}{2}
\tightlist
\item
  \textbf{Identify thematic strings in a paragraph from your manuscript.}
\end{enumerate}

Pick a paragraph from your manuscript that you think ``hangs together''. Do the same
exercise for this paragraph and identify the thematic strings.

Next pick a paragraph from your manuscript that feels vague when you re-read it.
Do the same exercise, and see if you can revise the paragraph so that its
thematic strings are consistent and clear.

\begin{enumerate}
\def\labelenumi{\arabic{enumi}.}
\setcounter{enumi}{3}
\tightlist
\item
  \textbf{Identify the issue in paragraphs from example papers.}
\end{enumerate}

Chapter 5 of \emph{Style} defines the \emph{issue} of a paragraph as the part that aims
``to put before the reader concepts or claims that the writer intends to expand
on in what follows.''

For the paragraph from prompt 2, the first sentence oseems to be its issue:

\begin{quote}
``Even if you use Twitter only for professional purposes, consider opening up a
little bit to show your followers you are a real person.''
\end{quote}

What theme is introduced at the end of this issue? Does the rest of the
paragraph follow through in talking about this theme? From the current version
of the issue sentence, is it easy to pick out the main concept that will be
developed in the rest of the paragraph (the discussion)?

Next, re-read the following paragraph from the Gall et al.~paper:

\begin{quote}
``A more subtle form of introducing hazard bias arises from issues of the
definition of the hazard and assigning loss estimates (by the original data
source) to predefined hazard categories within a database. This is most apparent
in the management of complex events involving multiple hazards versus a singular
hazard event. A tornado spawned by a hurricane is counted as a unique tornado
event, but it could also be lumped together within the entire hurricane event,
or both. Each loss database classifies events differently, especially when they
involve multiple hazard types (Guha-Sapir and Below 2002). Inconsistent naming
conventions and classification methodologies aggravate this problem and can
result in different (and/ or artificial) hazard categories for similar, if not
identical events. For example, Downton et al.~(2005) reveal a \$520 million''flood" loss in FEMA's database that was not in the NWS data. The discrepancy is
a result of differences in how each agency defines what constitutes a flood
event. In this case, the event (storm surge) was outside NWS's definition of a
flood."
\end{quote}

Highlight the text that gives the issue in this paragraph. What theme is identified
at the end of the issue sentence? Does the rest of the paragraph follow through
on this theme?

In the following version of the paragraph, I have rewritten the first sentence.
Does this version of the paragraph seem more or less coherent that the original
version? Identify the differences in the structure of the two versions of the
sentence that contribute to coherence of the paragraph.

\begin{quote}
``\textbf{More subtly, hazard bias arises because databases differ in how
they define disasters and sort hazards into predefined categories.} This is
most apparent in the management of complex events involving multiple hazards
versus a singular hazard event. A tornado spawned by a hurricane is counted as a
unique tornado event, but it could also be lumped together within the entire
hurricane event, or both. Each loss database classifies events differently,
especially when they involve multiple hazard types (Guha-Sapir and Below 2002).
Inconsistent naming conventions and classification methodologies aggravate this
problem and can result in different (and/ or artificial) hazard categories for
similar, if not identical events. For example, Downton et al.~(2005) reveal a
\$520 million''flood" loss in FEMA's database that was not in the NWS data. The
discrepancy is a result of differences in how each agency defines what
constitutes a flood event. In this case, the event (storm surge) was outside
NWS's definition of a flood."
\end{quote}

\begin{enumerate}
\def\labelenumi{\arabic{enumi}.}
\setcounter{enumi}{4}
\tightlist
\item
  \textbf{Identify the issue in paragraphs from your manuscript.}
\end{enumerate}

Pick a series of four or five paragraphs in your manuscript. For each paragraph,
highlight the sentence that states the issue. What concept is presented at the end
of that sentence? Highlight places in the rest of that paragraph that
follow up on that theme. Are there any thematic strings in the rest of the
paragraph that weren't ``announced'' in the issue sentence?

Re-write the issue sentence of each paragraph. As you do, keep in mind the
advice from the ``Sentences'' section on characters and actions. Also, make sure
that you place the theme you want to develop at the end of the issue sentence.

\begin{enumerate}
\def\labelenumi{\arabic{enumi}.}
\setcounter{enumi}{5}
\tightlist
\item
  \textbf{Identify the issue in every paragraph of your manuscript.}
\end{enumerate}

Now go through your entire manuscript and highlight the issue sentence in each
paragraph.

This should help highlight the structure of your paper and the major ideas you
are trying to share. Do your issues agree with what your goals for this paper?
Do they align with the ``Challenge'' at the end of your Introduction?

Make two lists: one with any issues that are in your paper but do not align with
your Introduction's Challenge and one with any points that you were hoping to
make in your manuscript but that aren't yet in the issue of a paragraph. Use
these two lists to mark paragraphs that you might consider cutting (or revising)
and to add ``placeholders'' for spots where you want to add paragraphs to address
issues you've missed.

\textbf{All the following steps are things for you to try after our workshop
meeting on ``Paragraphs''.}

\begin{enumerate}
\def\labelenumi{\arabic{enumi}.}
\setcounter{enumi}{6}
\tightlist
\item
  \textbf{Diagnose and revise a paragraph from an example paper.}
\end{enumerate}

Re-read the following paragraph from the Schwartz paper. What are the main characters
in this paragraph (revisit Chapter 2 of \emph{Style} if you need a refresher)? What are
the topic strings (revisit Chapter 3 of \emph{Style} for a description)? What
are the thematic strings? Once you have written down each of these, revise
the paragraph to make it more focused and easier to read.

\begin{quote}
``The question of disease etiology is complex. It is likely that a multitude of
causes is involved in the development of any particular disease. Where in the
causal chain, among the myriad of variables, one chooses to examine and
ascertain causation is often a question of where intervention is most
efficacious. That, in turn, is often a political and not a scientific issue. An
examination of the full range of variables potentially involved in disease
etiology, with a synthesis of findings from all levels of analysis, provides the
best opportunity for a full understanding of disease etiology.''
\end{quote}

\textbf{Try yourself before reading the next paragraph!} Here's my take. If I were
really working on this paper, I'd probably go make to the original author
with this paragraph to make sure that I haven't changed any of what she meant
to say with the edits.

\begin{quote}
``It is hard to figure out what causes a disease. Often, many causes
contribute, through a complex causal chain. When researchers try to clarify
this causal chain, their research is often framed by the question: which
interventions work best? However, an intervention succeeds or fails
as much from politics as from science. Research should instead\\
consider all the factors that might contribute to the disease's causal
chain, and when results are available at different levels of analysis,
they should be synthesized. Doing so provides our best opportunity
to fully understand disease etiology.''
\end{quote}

\begin{enumerate}
\def\labelenumi{\arabic{enumi}.}
\setcounter{enumi}{7}
\tightlist
\item
  \textbf{Revise the issue sentence in each paragraph of your manuscript.}
\end{enumerate}

Using what we've covered in ``Sentences'' and today, revise all the issue
sentences in your manuscript. Make sure that you place the concept that you want
to discuss in the paragraph at the end of the issue sentence.

\begin{enumerate}
\def\labelenumi{\arabic{enumi}.}
\setcounter{enumi}{8}
\tightlist
\item
  \textbf{Read Chapter 6 of \emph{Style} (online version).}
\end{enumerate}

This chapter continues to discuss how paragraphs can be made more coherent.
Think about the content in this chapter in terms of which paragraphs of your
manuscript have a point that comes at the end of the paragraph rather than
at the end of the paragraph's issue.

For the paragraph where you state your paper's ``Challenge'', this will almost
always be the case. Are there other paragraphs in your manuscript where this is
also the case? If this is not the case for the last paragraph of your
Introduction, do you think that paragraph would be more effective if you moved
the point to the end?

\emph{Additional reading.} If you would like more information on the ideas in this
chapter of \emph{Style}, you can also check out Chapter 11 from \emph{Writing Science}.
As a follow-up, you might also want to check out Chapter 11 from \emph{Writing Science
in Plain English}, which discusses how to arrange paragraphs.

\hypertarget{day-4-arguments}{%
\chapter{Day 4: Arguments}\label{day-4-arguments}}

\newthought{Today, we'll move back} to content. We'll talk about how to
present and support arguments in your manuscript. By ``arguments'', I don't
necessarily mean statements that are likely to cause strong disagreement, but
rather points that we're trying to convince the readers of. In this sense, the
``arguments'' we'll talk about today have a lot to do with the paragraph ``points''
covered in Chapter 6 of \emph{Style} (which you'll read after our Day 3 meeting).

\emph{The Craft of Research} has a whole section on presenting and building arguments
in a paper. This book is meant more for students who are writing reports or
theses, rather than people writing scientific manuscripts. However, it has a
whole section that usefully presents key ideas behind making arguments in your
writing. I have found it very useful to revisit these ideas when I feel like I'm
not making a point well in a manuscript.

The ideas we'll talk about today fall into the general category of \emph{rhetoric}.
We don't teach this as much as we should. We'll only get a taste today, but it's
a topic worth exploring more as you continue to work on writing and other forms
of communication to share your science.

\begin{enumerate}
\def\labelenumi{\arabic{enumi}.}
\tightlist
\item
  \textbf{Read Chapter 7 and 8 of \emph{The Craft of Research}.}
\end{enumerate}

These chapters give an overview of making arguments in writing and then
explain how to make claim.

\begin{enumerate}
\def\labelenumi{\arabic{enumi}.}
\setcounter{enumi}{1}
\tightlist
\item
  \textbf{Identify claims in the example papers.}
\end{enumerate}

Re-read the two first paragraphs in the Chelygina et al.~paper. Highlight
a single claim that you think is the most important claim made in this
section:

\begin{quote}
"Twitter is one of the most popular social media platforms, with over 320
million active users as of February 2019. Twitter users can enjoy free content
delivered by other users whom they actively decide to follow. However, unlike in
other areas where Twitter is used passively (e.g., to follow influential figures
and/or information agencies), in science it can be used in a much more active,
collaborative way: to ask for advice, to form new bonds and scientific
collaborations, to announce jobs and find employees, to find new mentors and
jobs. This is particularly important in the early stages of a scientific career,
during which lack of collaboration or delayed access to information can have the
most impact.
\end{quote}

\begin{quote}
``For these reasons, using Twitter appropriately {[}1{]} can be more than
just a social media activity; it can be a real career incubator in which
researchers can develop their professional circles, launch new research projects
and get helped by the community at various stages of the projects. Twitter is a
tool that facilitates decentralization in science; you are able to present
yourself to the community, to develop your personal brand, to set up a dialogue
with people inside and outside your research field and to create or join
professional environment in your field without mediators such as your direct
boss.''
\end{quote}

Re-read the following paragraph in the Gall et al.~paper. Highlight
a single claim that you think is the most important claim in this paragraph:

\begin{quote}
``A more subtle form of introducing hazard bias arises from issues of the
definition of the hazard and assigning loss estimates (by the original data
source) to predefined hazard categories within a database. This is most apparent
in the management of complex events involving multiple hazards versus a singular
hazard event. A tornado spawned by a hurricane is counted as a unique tornado
event, but it could also be lumped together within the entire hurricane event,
or both. Each loss database classifies events differently, especially when they
involve multiple hazard types (Guha-Sapir and Below 2002). Inconsistent naming
conventions and classification methodologies aggravate this problem and can
result in different (and/ or artificial) hazard categories for similar, if not
identical events. For example, Downton et al.~(2005) reveal a \$520 million''flood" loss in FEMA's database that was not in the NWS data. The discrepancy is
a result of differences in how each agency defines what constitutes a flood
event. In this case, the event (storm surge) was outside NWS's definition of a
flood."
\end{quote}

Page 123 of \emph{The Craft of Research} gives several categories for types of
claims. What type of claim do you think is being made in each of the two
paragraphs above? Do you think each is more conceptual or more practical?
Does this agree with the surrounding content of the paper (i.e., is it trying
to change the readers mind or trying to make the reader do something)?
Do either of the claims include a hedge or qualification?

\begin{enumerate}
\def\labelenumi{\arabic{enumi}.}
\setcounter{enumi}{2}
\tightlist
\item
  \textbf{Identify and revise a claim in your manuscript.}
\end{enumerate}

Pick a paragraph of your manuscript where you are making an important claim.
Highlight the sentence that makes the key claim in that paragraph.

Based on Chapter 8 of \emph{The Craft of Research}, describe and assess this claim.
What type of claim is it (see p.~123)? Is it practical or conceptual? Is it
too weak or too strong, given what you think you can reasonable claim based on
your evidence and logic? Is the claim specific? Is your claim significant?

Revise the claim based on your assessment. If you are trying to get the readers
to do something with your paper, make sure it is a claim of action. If you
are trying to change readers minds, make sure it is a conceptual claim. If it
seems too weak, see if there are any hedges or qualifications that should be
taken out. If it instead seems too strong, consider adding these.

\begin{enumerate}
\def\labelenumi{\arabic{enumi}.}
\setcounter{enumi}{3}
\tightlist
\item
  \textbf{Read Chapters 9 and 10 of \emph{The Craft of Research}.}
\end{enumerate}

These chapters describe how to build your argument through evidence and reasons
to support your claim, as well as qualifications and acknowledgments when
needed.

\begin{enumerate}
\def\labelenumi{\arabic{enumi}.}
\setcounter{enumi}{4}
\tightlist
\item
  \textbf{Identify evidence, reasons, acknowledgments in example papers.}
\end{enumerate}

Re-read the two first paragraphs in the Chelygina et al.~paper:

\begin{quote}
"Twitter is one of the most popular social media platforms, with over 320
million active users as of February 2019. Twitter users can enjoy free content
delivered by other users whom they actively decide to follow. However, unlike in
other areas where Twitter is used passively (e.g., to follow influential figures
and/or information agencies), in science it can be used in a much more active,
collaborative way: to ask for advice, to form new bonds and scientific
collaborations, to announce jobs and find employees, to find new mentors and
jobs. This is particularly important in the early stages of a scientific career,
during which lack of collaboration or delayed access to information can have the
most impact.
\end{quote}

\begin{quote}
``For these reasons, using Twitter appropriately {[}1{]} can be more than
just a social media activity; it can be a real career incubator in which
researchers can develop their professional circles, launch new research projects
and get helped by the community at various stages of the projects. Twitter is a
tool that facilitates decentralization in science; you are able to present
yourself to the community, to develop your personal brand, to set up a dialogue
with people inside and outside your research field and to create or join
professional environment in your field without mediators such as your direct
boss.''
\end{quote}

Based on the key claim you identified earlier for these paragraphs, write down
the reasons that the text gives to support that claim. Write these down by
first writing the claim in bold and then providing a bulleted list with
each key reason they give in support of that claim (you can put these in
the order they're given in the text, but you don't have to).

Assess these reasons. What evidence do they give in support of these reasons?
Are there any where they assume that no evidence is needed, because the reason
is commonly accepted as valid? Can you identify any qualifications /
acknowledgments that are included in their argument in support of the key claim
or any of the reasons they provide for the main claim?

Next answer the same questions for the paragraph from the Gall et al.~paper:

\begin{quote}
``A more subtle form of introducing hazard bias arises from issues of the
definition of the hazard and assigning loss estimates (by the original data
source) to predefined hazard categories within a database. This is most apparent
in the management of complex events involving multiple hazards versus a singular
hazard event. A tornado spawned by a hurricane is counted as a unique tornado
event, but it could also be lumped together within the entire hurricane event,
or both. Each loss database classifies events differently, especially when they
involve multiple hazard types (Guha-Sapir and Below 2002). Inconsistent naming
conventions and classification methodologies aggravate this problem and can
result in different (and/ or artificial) hazard categories for similar, if not
identical events. For example, Downton et al.~(2005) reveal a \$520 million''flood" loss in FEMA's database that was not in the NWS data. The discrepancy is
a result of differences in how each agency defines what constitutes a flood
event. In this case, the event (storm surge) was outside NWS's definition of a
flood."
\end{quote}

\begin{enumerate}
\def\labelenumi{\arabic{enumi}.}
\setcounter{enumi}{5}
\tightlist
\item
  \textbf{Identify evidence, reasons, and acknowledgments in example papers.}
\end{enumerate}

Use the paragraph from your manuscript that you used for prompt 3. Write out
the key claim and the reasons you give to support it, using the same style as
for the last prompt.

Evaluate these reasons using the same questions as in the last prompt. What
evidence do you give in support of these reasons? Are there any where you assume
that no evidence is needed, because the reason is commonly accepted as valid?
Are there any qualifications / acknowledgments that are included in their
argument in support of the key claim or any of the reasons you provide for the
main claim?

Revise the paragraph based on your assessment. If there are areas where you need
more reasons or evidence, you can put placeholders in to note what's missing and
do more research later to fill those in.

\begin{enumerate}
\def\labelenumi{\arabic{enumi}.}
\setcounter{enumi}{6}
\tightlist
\item
  \textbf{Read Chapter 11 of \emph{The Craft of Research}.}
\end{enumerate}

This chapter explains when and how to include warrants in your arguments, to
connect reasons to claims. In student's writing, there are often too few warrants.
I think this might be because students often think that their reader will know
at least as much as they do about the topic, and so if the connection is obvious
to the student it must also be to the reader.

Remember that this is no longer the case. For your scientific manuscripts, you
will be the expert in the room. You will know more about the very specific topic
than almost all the readers, possibly including the editor and reviewers for the
peer review process. Connections that seem obvious to you will not have occurred
to many of them.

\begin{enumerate}
\def\labelenumi{\arabic{enumi}.}
\setcounter{enumi}{7}
\tightlist
\item
  \textbf{Add a warrant in an example paper.}
\end{enumerate}

In the Chelygina et al.~paper, the claim is made that:

\begin{quote}
``Using Twitter appropriately \ldots{} can be a real career incubator''
\end{quote}

One of the reasons that they give is:

\begin{quote}
``In science {[}Twitter{]} can be used in a much more active,
collaborative way: to ask for advice, to form new bonds and scientific
collaborations, to announce jobs and find employees, to find new mentors and
jobs.''
\end{quote}

To connect this reason with their claim, they provide the following warrant:

\begin{quote}
``This is particularly important in the early stages of a scientific career,
during which lack of collaboration or delayed access to information can have the
most impact.''
\end{quote}

Explain how this warrant connects this reason to the key claim they make.

Another of the reasons that they give, but for which they do not provide a
warrant, is:

\begin{quote}
``Twitter users can enjoy free content delivered by other users whom they
actively decide to follow.''
\end{quote}

Rewrite the first two paragraphs of their introduction to add a warrant
connecting this reason to the main claim. Do you think this added warrant is
necessary, or will it be clear to most readers how they connect without it?

\begin{enumerate}
\def\labelenumi{\arabic{enumi}.}
\setcounter{enumi}{8}
\tightlist
\item
  \textbf{Add warrants to your manuscript.}
\end{enumerate}

Revisit the paragraph from your manuscript that you used for prompts 3 and 6.
Do you give a warrant for every reason that you provide for your key claim
in that paragraph? If not, pick the warrant-less reason for which you think the
connection will be the least obvious for reader and add a warrant.

In re-reading the paragraph, do you think this warrant is useful or unnecessary?
Skim through other paragraphs in your manuscript and find 2--3 other places
where a warrant might be helpful and add one.

\textbf{All the following steps are things for you to try after our workshop
meeting on ``Arguments''.}

\begin{enumerate}
\def\labelenumi{\arabic{enumi}.}
\setcounter{enumi}{9}
\tightlist
\item
  \textbf{Identify and revise more claims in your manuscript.}
\end{enumerate}

Repeat prompt 3 for more paragraphs in your manuscript. Once you have assessed
and revised more of your claims, highlight all of the key claims that you are
making in the manuscript. This will help in assessing the overall structure of
the manuscript and how solid your key arguments are.

For the paragraphs or sections with the most important of these claims, repeat
prompt 6 to diagnose and fix any problems in your argument for the claim.

\begin{enumerate}
\def\labelenumi{\arabic{enumi}.}
\setcounter{enumi}{10}
\tightlist
\item
  \textbf{Re-read Chapters 8 and 9 of \emph{Writing Science}.}
\end{enumerate}

These chapters talk about how to write the middle and end of a scientific
manuscript, which the author calls the ``Action'' and the ``Resolution''.
At a high level, your manuscript makes its main claim through its
Challenge at the end of the Opening. This is where it claims that
it will address an important problem. The rest of the paper works to
build the argument for this claim by providing evidence, warrents,
qualifications, etc.

\begin{enumerate}
\def\labelenumi{\arabic{enumi}.}
\setcounter{enumi}{11}
\tightlist
\item
  \textbf{Diagnose for your manuscript if the Action and Resolution support the claim made by the Challenge.}
\end{enumerate}

Yesterday, you identified the issues in each of the paragraphs in your
manuscript. Think about how, at a broad level, these issues contribute to the
``claim'' that the Challenge of your paper makes. Does your Challenge agree in
topic and scope with what's written in the rest of your manuscript? If not, do
you think that you should revise the Challenge, or would it be better to edit
the rest of the manuscript to better support the Challenge? Can you identify
places where your current argument is weak in the paper as a whole?

\begin{enumerate}
\def\labelenumi{\arabic{enumi}.}
\setcounter{enumi}{12}
\tightlist
\item
  \textbf{Read Chapter 18 of \emph{Writing Science}.}
\end{enumerate}

This chapter describes how to discuss limitations of your research in a
manuscript. In \emph{The Craft of Research}, we read about the importance of
qualifications. In a scientific manuscript with original research, some of your
``arguments'' will be the interpretation you give for your results. Some of the
qualifications/acknowledgments for these arguments will come through your
discussion of the limitations of the study.

\begin{enumerate}
\def\labelenumi{\arabic{enumi}.}
\setcounter{enumi}{13}
\tightlist
\item
  \textbf{Draft text on limitations for your study.}
\end{enumerate}

Think about your own manuscript. Does it describe original research for which
you should describe study limitations? (This may not be the case if you
manuscript is a commentary or editiorial.) If so, what are some of the key
limitations of the study? Think of these in terms of things that you would fix
if it were practical.

Draft some text that first describes each of these limitations and then explains
why you didn't avoid them in the research (for example, the alternative could
not practically be done, or they were unlikely to influence the results much).

If your paper does not cover original research, think of some other high-level
qualifications that it may be important to make in your paper. It may be helpful
to think of which comments you're most worried about getting from reviewers.

\begin{enumerate}
\def\labelenumi{\arabic{enumi}.}
\setcounter{enumi}{14}
\tightlist
\item
  \textbf{Read Chapters 13, 15, and 16 of \emph{The Craft of Research}.}
\end{enumerate}

These chapters give more guidance on how to put all these pieces together in
writing. They include advice on how to present evidence visually, through tables
and figures in the paper, as well as how to reorganize and edit large parts of
your text, especially the Introduction and Conclusions, to improve arguments.

\hypertarget{day-5-words}{%
\chapter{Day 5: Words}\label{day-5-words}}

\newthought{We'll wrap up by talking about} editing you can do at the level
of words. I find this to be the easiest and most fun type of editing. A lot of
it doesn't take a lot of mental work, but your writing immediately feels much
lighter!

A lot of the advice in this section will have a secondary benefit---it will
reduce the number of words it takes you to say something. In scientific writing,
we often have to fit our final work within a constrained word count or page limit.
Since we want to have high-impact, high-content papers, this can be a big
struggle. The suggestions today will help with this.

The other key point for today is how word choices can make your writing
easier to read and understand. There are two key angles here. First, you want
to identify the ``terms of art'' in your paper. What are the technical terms
that mean something very specific, and that you can't simplify without losing
important subtlety? For these words, you must make sure that (1) you're using
the right term for your field and (2) you consistently use the same term for that
idea. Second, you have all the other words in your paper, the ones that aren't
terms of art. Scientific reading is hard enough just because of the scientific
content. Therefore, I feel strongly that you should try to make all these other
words---the ones that aren't terms of art---as simple as possible. Therefore,
part of the editing process at the word level will be to simplify any vocabulary
that doesn't absolutely have to be more complex.

The techniques we'll cover today are outlines in several of the books---\emph{Style} has
a chapter on ``Concision'' that we'll cover, and \emph{Writing Science in Plain English} and
\emph{Writing Science} both have helpful chapters.

\begin{enumerate}
\def\labelenumi{\arabic{enumi}.}
\tightlist
\item
  **Read Chapter 7 from \emph{Style} (same chapter in the second edition) and Chapter 6 from *Writing Science in Plain English.**
\end{enumerate}

These two chapters cover how to remove needless words from your writing. Both cover
several ways that words can be redundant.

\emph{Additional reading.} If you would like more information on the ideas in these
chapters of \emph{Style}, you can also check out Chapter 16 from \emph{Writing Science}.

\begin{enumerate}
\def\labelenumi{\arabic{enumi}.}
\setcounter{enumi}{1}
\tightlist
\item
  \textbf{Diagnose and fix word redundance in paragraphs from example papers.}
\end{enumerate}

On an earlier day, we looked closesly at the following paragraph from the Schwartz
paper:

\begin{quote}
``The question of disease etiology is complex. It is likely that a multitude of
causes is involved in the development of any particular disease. Where in the
causal chain, among the myriad of variables, one chooses to examine and
ascertain causation is often a question of where intervention is most
efficacious. That, in turn, is often a political and not a scientific issue. An
examination of the full range of variables potentially involved in disease
etiology, with a synthesis of findings from all levels of analysis, provides the
best opportunity for a full understanding of disease etiology.''
\end{quote}

Based on what you just read, identify words in this text that might be redundant.
Try re-writing the text to remove those redundant words. You can either keep the
same structure for the paragraph otherwise, or you can start from the re-written
version of this paragraph you created on an earlier day.

Do the same exercise with the following paragraph from the Cheplygina et al.~paper:

\begin{quote}
``But even more importantly, Twitter culture has exposed a part of academia
that has traditionally always been hidden from view, namely the inception of new
research activities. Now, ECRs can observe and even join the process of creating
national or international research projects (for instance, {[}6{]} stems from a
discussion at Twitter and Bik et al {[}1{]} write their work resulted from online
interactions). Senior researchers openly share ideas through Twitter and this
can lead to the development of new concepts which often move on to become
fully-fledged research projects.''
\end{quote}

\begin{enumerate}
\def\labelenumi{\arabic{enumi}.}
\setcounter{enumi}{2}
\tightlist
\item
  \textbf{Diagnose and fix redundant words in your manuscript.}
\end{enumerate}

Pick a string of three or four paragraphs in your manuscript. Go carefully through
each type of redundancy described in the reading from prompt 1. Identify any
cases in your paragraphs and fix them.

For some of these, you can use the ``Search'' tool to search for potentially redundant
words. For example, you can do a search for meaningless words like ``actually'',
``really'', and ``various''. To look for doubled words, it might be helpful to do a
search for ``and''. If you're struggling with some of the other types of redundancies,
you might want to look through some of the many online lists with examples,
including: \url{https://www.dailywritingtips.com/50-redundant-phrases-to-avoid/},
\url{https://forge.medium.com/close-proximity-end-result-and-more-redundant-words-to-delete-from-your-writing-3258be693a3d},
\url{https://kathysteinemann.com/Musings/redundant/}, and
\url{https://www.thoughtco.com/common-redundancies-in-english-1692776}.

\begin{enumerate}
\def\labelenumi{\arabic{enumi}.}
\setcounter{enumi}{3}
\tightlist
\item
  \textbf{Read Chapter 15 from \emph{Writing Science} and Chapter 5 from \emph{Writing Science in Plain English}.}
\end{enumerate}

These two chapters cover how to choose the word to use for words that must
stay---the ones that aren't redundant. They cover two important ideas. First,
there are some words that mean something very specific in your field, and that
provide an immediate way to get a specific and often subtle idea across to
readers who have been trained in your field. For these words, you need to make
sure you are using the right term, and you don't want to simplify them because
they lose their precise and subtle meaning.

However, typically these words will be only a minor proportion of all the words
in your manuscript. It is easy for jargon or unneccessarily fancy words to creep
in everywhere else. Again, you are writing to describe ideas and findings that
are very complex. It will be hard for readers to process those ideas. To help
them a little, please, please, please try to make everything surrounding and
supporting those ideas as simple and clear as you can. If you worry that your
edited version of the manuscript is going to make you ``sound dumb'' because it
sounds so simple, then you are on the right track!

Also---and this is something that I continue to feel when I edit my own
papers---when you edit to clarify and simplify, you may sometimes feel exposed
or unprotected, because your idea is just out there for anyone to understand,
not in the shadow behind fancy words or sentence constructions. If you've edited
your manuscript well, you should feel a bit exposed. You're giving people a look
right into your head, and they'll be able to evaluate your ideas for what they
are. This is a normal feeling, and it feels scary, but it means you're on to
something good. If the idea itself is bad, you'll now see that more clearly and
be able to fix it. If it's a good idea, then other people will be able to grasp
it.

\begin{enumerate}
\def\labelenumi{\arabic{enumi}.}
\setcounter{enumi}{4}
\tightlist
\item
  \textbf{Identify technical terms versus jargon in a paragraph from an example paper.}
\end{enumerate}

Re-read the following paragraph from the Gall et al.~paper:

\begin{quote}
``Natural hazard losses exhibit an upward trend over time (Fig. 2). This is a
function of increases in wealth and population (Cutter and Emrich 2005; Pielke
et al.~2008) but is also attributed to better loss accounting in recent years.
The escalating pattern of hazard losses is therefore partially an artifact of
advances in reporting losses, but how much or how little this effect contributes
to the skyrocketing losses in comparison to effects of population growth and
increasing wealth in high hazard areas is unclear.''
\end{quote}

Make a list of the terms in this paragraph that you think are technical terms,
as defined on page 147 of \emph{Writing Science}. Then, highlight any terms that you
think might be jargon, again using the definition from \emph{Writing Science}.

For each term in the ``technical terms'' list, write down what you think it's
specific definition is. Look earlier in the paper. Is the term specifically
defined or explained by the authors earlier in the manuscript?

Finally, revise the paragraph to replace all the jargon terms with simpler language.

\begin{enumerate}
\def\labelenumi{\arabic{enumi}.}
\setcounter{enumi}{5}
\tightlist
\item
  \textbf{Identify technical terms necessary in your manuscript.}
\end{enumerate}

Go through your entire manuscript and make a glossary of the technical terms
that you \emph{must} use in the manuscript. These might be proper nouns, like the
name of a specific package repository like ``CRAN'' or a database like ``NOAA Storm
Events''. They may be ideas that are very specific to your topic, like
``reproducibility'' or ``exposure misclassification''. In your glossary, include
both the term and your definition for it.

\begin{enumerate}
\def\labelenumi{\arabic{enumi}.}
\setcounter{enumi}{6}
\tightlist
\item
  \textbf{Diagnose and fix jargon in your manuscript.}
\end{enumerate}

Use the same paragraphs from your manuscript that you did for prompt 3.
Highlight any terms in these paragraphs that are in the glossary you created in
the last prompt. Now look at all the remaining words. Are any more complex than
they need to be? Identify jargon and revise the paragraphs to use simpler words.

\hypertarget{wrap-up}{%
\chapter{Wrap-up}\label{wrap-up}}

\newthought{If you're doing it right,} you will continue working to improve your
writing your whole career. We can only cover a bit in one week. Here are some tips for
things to do to continue:

\begin{enumerate}
\def\labelenumi{\arabic{enumi}.}
\tightlist
\item
  \textbf{Read a lot.} Ultimately, you will have to read a lot of journal articles in your
  field to internalize the style of your field. By ``style'' here, I mean things like
  what the Introduction normally covers (and in what depth), how limitations of the
  study are discussed, how much detail should be included in the Methods, and how
  materials are split between the main text and the Supplemental Online Material for
  manuscripts. As a student, you will need to read a lot anyway to become familiar with
  the literature. In later stages of your career, you will read a lot (of varying quality)
  through service roles, like as a peer reviewer. At all stages, you will learn more about
  what works and what doesn't, what's expected and what isn't, in your field if you read
  a lot of journal articles in that field.
\item
  \textbf{Find some favorite authors.}
  For environmental epidemiology, one of my favorite authors is Ben Armstrong. He can
  explain very complex methodological approaches very clearly. For R programming, I love
  reading things that have been written by Jenny Bryan and Richie Cotton.
\item
  \textbf{Work on grammar and usage or find a good copyeditor.}
\item
  \textbf{Try out writing groups and sessions.}
\item
  \textbf{Explore brainstorming, planning, and organizing tools for writing.}
\item
  \textbf{Learn LaTeX.}
\item
  \textbf{Find your favorite text editor and learn to use it well.}
\item
  \textbf{Learn more about rhetoric.}
\end{enumerate}

\bibliography{book.bib,packages.bib}



\end{document}
